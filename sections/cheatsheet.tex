\appendix
\chapter{Exam Quick Reference}

\section{C Operator Precedence}

% The [H] specifier forces the image to stay here (requires \usepackage{float})
\begin{figure}[H]
    \centering
    % Ensure 'operator_precedence.png' is inside your 'images/' folder
    \includegraphics[width=0.95\textwidth]{operator-precedence.png} 
    \caption{C Operator Precedence and Associativity Table. (Note: Unary operators (right-to-left) bind tighter than arithmetic.)}
    \label{fig:operator_precedence}
\end{figure}

\begin{specbox}{Exam Reminder: Associativity}
    Remember that assignment operators (\texttt{=, +=, -=}) associate \textbf{Right-to-Left}.
    \begin{itemize}
        \item \texttt{a = b = c = 0;} is valid and executes as \texttt{a = (b = (c = 0));}
        \item \texttt{a / b / c} executes as \texttt{(a / b) / c} (Left-to-Right).
    \end{itemize}
\end{specbox}

\section{Bitwise Cheat Sheet}
\begin{itemize}
    \item \textbf{Set Bit 3:} \texttt{REG |= (1 << 3);}
    \item \textbf{Clear Bit 3:} \texttt{REG \&= \textasciitilde(1 << 3);}
    \item \textbf{Toggle Bit 3:} \texttt{REG \textasciicircum= (1 << 3);}
    \item \textbf{Check Bit 3:} \texttt{if (REG \& (1 << 3))}
\end{itemize}

\section{Common Register Maps}
% You can paste your most used memory addresses here
\begin{lstlisting}[language=C]
#define GPIOA_MODER  (*((volatile uint32_t *) 0x48000000))
#define RCC_AHB1ENR  (*((volatile uint32_t *) 0x40023830))
\end{lstlisting}